\documentclass[12pt]{article}
\usepackage{amsmath,amssymb,amsthm}
\usepackage{geometry}
\usepackage{hyperref}
\usepackage{algorithm}
\usepackage{algpseudocode}
\geometry{margin=1in}

\newtheorem{definition}{Definition}
\newtheorem{theorem}{Theorem}

\title{Election Forecasting Models:\\Mathematical Foundations}
\author{State-Level Presidential Election Forecasting}
\date{November 2024}

\begin{document}
\maketitle

\begin{abstract}
We present four probabilistic models for forecasting U.S. presidential elections using state-level polling data: (1) Poll Average, a simple weighted baseline; (2) Kalman Diffusion, treating vote margins as Brownian motion with drift; (3) Improved Kalman, adding stronger regularization; and (4) Hierarchical Bayes, combining fundamentals priors with systematic bias correction. We derive the mathematical foundations for each approach and evaluate their performance on the 2016 election.
\end{abstract}

\tableofcontents

\section{Problem Setup and Notation}

\subsection{Data Structure}

For each state $s \in \{1, \ldots, S\}$ (typically $S=51$ including DC), we observe:

\begin{itemize}
    \item $N_s$ polls indexed by $i = 1, \ldots, N_s$
    \item Poll $i$ has:
    \begin{itemize}
        \item Midpoint date $t_i \in [t_0, T_{elec}]$
        \item Sample size $n_i$
        \item Democratic votes $d_i$, Republican votes $r_i$
        \item Two-party margin $y_i = \frac{d_i - r_i}{d_i + r_i} \in [-1, 1]$
        \item Pollster $p(i) \in \mathcal{P}$
    \end{itemize}
\end{itemize}

\subsection{Target Quantity}

On election day $T_{elec}$, the actual two-party margin is:
\begin{equation}
m_s^{true} = \frac{D_s - R_s}{D_s + R_s}
\end{equation}
where $D_s$, $R_s$ are total votes received.

\subsection{Forecasting Task}

Given polls up to forecast date $t_{forecast} < T_{elec}$, predict:
\begin{enumerate}
    \item \textbf{Win probability}: $P(m_s^{true} > 0 \mid \text{data up to } t_{forecast})$
    \item \textbf{Expected margin}: $\mathbb{E}[m_s^{true} \mid \text{data}]$
    \item \textbf{Uncertainty}: Standard deviation $\sigma_s$
\end{enumerate}

\subsection{Evaluation Metrics}

For a set of predictions $\{(p_s, \hat{m}_s)\}_{s=1}^S$ and outcomes $\{(o_s, m_s^{true})\}_{s=1}^S$ where $o_s = \mathbb{1}\{m_s^{true} > 0\}$:

\textbf{Brier Score} (calibration of probabilities):
\begin{equation}
BS = \frac{1}{S} \sum_{s=1}^S (p_s - o_s)^2 \in [0,1]
\end{equation}

\textbf{Log Loss} (penalizes overconfidence):
\begin{equation}
LL = -\frac{1}{S} \sum_{s=1}^S \left[ o_s \log(p_s + \epsilon) + (1-o_s) \log(1-p_s+\epsilon) \right]
\end{equation}
where $\epsilon = 10^{-10}$ prevents numerical issues.

\textbf{Mean Absolute Error} (point prediction accuracy):
\begin{equation}
MAE = \frac{1}{S} \sum_{s=1}^S |\hat{m}_s - m_s^{true}|
\end{equation}

Lower is better for all three metrics.

\section{Model 1: Poll Average (Baseline)}

\subsection{Motivation}

The simplest approach: average recent polls, weighted by sample size. This represents what a journalist would do manually without statistical modeling.

\subsection{Algorithm}

\begin{algorithm}
\caption{Poll Average Forecasting}
\begin{algorithmic}[1]
\State \textbf{Input:} Polls $\{(t_i, y_i, n_i)\}$ up to $t_{forecast}$, days to election $D$
\State \textbf{Parameters:} Window $W = 14$ days
\State
\State Filter polls: $\mathcal{I} = \{i : t_i \geq t_{forecast} - W\}$
\State
\If{$|\mathcal{I}| < 3$}
    \State Use last 5 polls instead
\EndIf
\State
\State Compute weights: $w_i = n_i$ for $i \in \mathcal{I}$, normalize: $w_i \leftarrow w_i / \sum_j w_j$
\State
\State Predicted margin: $\hat{m} = \sum_{i \in \mathcal{I}} w_i y_i$
\State
\State Empirical std: $s = \sqrt{\sum_{i \in \mathcal{I}} (y_i - \hat{m})^2 / (|\mathcal{I}| - 1)}$
\State
\State Average sample size: $\bar{n} = \sum_{i \in \mathcal{I}} w_i n_i$
\State
\State Sampling std: $\sigma_{samp} = 1 / \sqrt{\bar{n}}$
\State
\State Base uncertainty: $\sigma_{base} = \max(s, \sigma_{samp}, 0.02)$
\State
\State Horizon uncertainty: $\sigma_{horizon} = 0.001 \cdot D$
\State
\State Total uncertainty: $\sigma = \sqrt{\sigma_{base}^2 + \sigma_{horizon}^2}$
\State
\State Win probability: $p = \Phi(\hat{m} / \sigma)$, clipped to $[0.05, 0.95]$
\State
\State \Return $(p, \hat{m}, \sigma)$
\end{algorithmic}
\end{algorithm}

\subsection{Mathematical Justification}

\textbf{Weighted average as MLE:}

Assume each poll $y_i$ is an independent draw:
\begin{equation}
y_i \sim \mathcal{N}(\mu, \sigma_i^2)
\end{equation}
where $\sigma_i^2 = \frac{1}{4n_i}$ (approximate sampling variance for a proportion).

The maximum likelihood estimator for $\mu$ with known variances is:
\begin{equation}
\hat{\mu}_{MLE} = \frac{\sum_i w_i y_i}{\sum_i w_i}, \quad w_i = \frac{1}{\sigma_i^2} = 4n_i
\end{equation}

For simplicity, we use $w_i = n_i$ (proportional weighting).

\textbf{Forecast horizon uncertainty:}

The term $\sigma_{horizon} = 0.001 \cdot D$ adds uncertainty that grows linearly with days until election. This captures:
\begin{itemize}
    \item Possible shifts in voter sentiment
    \item Systematic polling errors
    \item Late-deciding voters
\end{itemize}

This linear scaling is implemented in the code as:
\begin{verbatim}
horizon_uncertainty = 0.001 * days_to_election
total_std = np.sqrt(total_std**2 + horizon_uncertainty**2)
\end{verbatim}

\subsection{Strengths and Weaknesses}

\textbf{Strengths:}
\begin{itemize}
    \item Simple and transparent
    \item No parameters to tune (besides window size)
    \item Robust to model misspecification
    \item Fast to compute
\end{itemize}

\textbf{Weaknesses:}
\begin{itemize}
    \item Ignores temporal structure (treats all polls in window equally)
    \item Doesn't model pollster house effects
    \item Arbitrary window size choice
    \item Wastes older polling data
\end{itemize}

\section{Model 2: Kalman Diffusion}

\subsection{Motivation}

Model the latent "true" vote margin $X_t$ as evolving continuously over time, observed through noisy polls. This allows us to:
\begin{itemize}
    \item Use all available polls (not just recent ones)
    \item Smooth out polling noise
    \item Account for pollster-specific biases
    \item Make principled forecasts with uncertainty quantification
\end{itemize}

\subsection{Model Specification}

\textbf{Latent state process} (Brownian motion with drift):
\begin{equation}
dX_t = \mu \, dt + \sigma \, dW_t
\end{equation}
where $W_t$ is standard Brownian motion.

In discrete time ($\Delta t = 1$ day):
\begin{equation}
X_{t+1} | X_t \sim \mathcal{N}(X_t + \mu, \sigma^2)
\end{equation}

\textbf{Parameters:}
\begin{itemize}
    \item $\mu$: daily drift (trend toward one candidate)
    \item $\sigma^2$: daily diffusion variance (day-to-day volatility)
\end{itemize}

\textbf{Observation model:}

Poll $i$ at time $t_i$ by pollster $p(i)$ observes:
\begin{equation}
y_i = X_{t_i} + b_{p(i)} + \eta_i
\end{equation}
where:
\begin{itemize}
    \item $b_p$: systematic bias (house effect) of pollster $p$
    \item $\eta_i \sim \mathcal{N}(0, \tau_i^2)$: measurement error
\end{itemize}

\textbf{Observation variance:}
\begin{equation}
\tau_i^2 = \frac{1}{n_i} + \tau_{extra}^2
\end{equation}
where $\tau_{extra}^2 = (0.015)^2$ accounts for methodology differences beyond sampling error.

\subsection{Parameter Estimation: EM Algorithm}

We estimate $\theta = \{\mu, \sigma^2, \{b_p\}_{p \in \mathcal{P}}\}$ using the Expectation-Maximization (EM) algorithm.

\subsubsection{E-step: Kalman Filter + RTS Smoother}

Given current parameters $\theta^{(k)}$, compute the posterior distribution of the latent path $\{X_t\}$ given observations $\{y_i\}$.

\textbf{State-space form:}
\begin{align}
X_t &= X_{t-1} + \mu \Delta t_t + \epsilon_t, \quad \epsilon_t \sim \mathcal{N}(0, \sigma^2 \Delta t_t) \label{eq:state}\\
y_t &= X_t + \nu_t, \quad \nu_t \sim \mathcal{N}(0, \tau_t^2) \label{eq:obs}
\end{align}

\textbf{Kalman filter (forward pass):}

Initialize: $X_{0|0} = y_0$, $P_{0|0} = \tau_0^2$ (observe first poll)

For $t = 1, \ldots, T$:
\begin{align}
\text{Predict:} \quad X_{t|t-1} &= X_{t-1|t-1} + \mu \Delta t_t \\
P_{t|t-1} &= P_{t-1|t-1} + \sigma^2 \Delta t_t \\
\text{Update:} \quad K_t &= \frac{P_{t|t-1}}{P_{t|t-1} + \tau_t^2} \quad \text{(Kalman gain)} \\
X_{t|t} &= X_{t|t-1} + K_t (y_t - X_{t|t-1}) \\
P_{t|t} &= (1 - K_t) P_{t|t-1}
\end{align}

\textbf{RTS smoother (backward pass):}

Initialize: $X_{T|T}^s = X_{T|T}$, $P_{T|T}^s = P_{T|T}$

For $t = T-1, \ldots, 0$:
\begin{align}
J_t &= \frac{P_{t|t}}{P_{t|t} + \sigma^2 \Delta t_{t+1}} \\
X_{t|T}^s &= X_{t|t} + J_t (X_{t+1|T}^s - X_{t|t} - \mu \Delta t_{t+1}) \\
P_{t|T}^s &= P_{t|t} + J_t^2 (P_{t+1|T}^s - P_{t|t} - \sigma^2 \Delta t_{t+1})
\end{align}

Output: smoothed estimates $\{X_{t|T}^s, P_{t|T}^s\}$ given all data.

\subsubsection{M-step: Parameter Updates}

\textbf{Update pollster biases:}

For each pollster $p$, with shrinkage $\lambda = 0.5$:
\begin{equation}
b_p^{new} = \begin{cases}
\lambda \cdot \frac{1}{N_p} \sum_{i: p(i)=p} (y_i - X_{t_i|T}^s) & \text{if } N_p \geq 2 \\
0 & \text{otherwise}
\end{cases}
\end{equation}

\textbf{Update drift:}
\begin{equation}
\mu^{new} = \frac{1}{\sum_{t=1}^{T-1} \Delta t_t} \sum_{t=1}^{T-1} (X_{t+1|T}^s - X_{t|T}^s)
\end{equation}

\textbf{Update diffusion variance:}
\begin{equation}
(\sigma^2)^{new} = \max\left( \frac{1}{\sum_{t=1}^{T-1} \Delta t_t} \sum_{t=1}^{T-1} \left( P_{t|T}^s + P_{t+1|T}^s \right), 0.0005 \right)
\end{equation}
with floor $0.0005$ to prevent degeneracy.

Iterate E and M steps for $K=10$ iterations.

\subsection{Forecasting}

Given the posterior at the last observation time $t_{last}$, forecast to election day $T_{elec}$ using Monte Carlo simulation.

\textbf{Incorporating fundamentals prior:}

Apply prior weight $w = 0.1$:
\begin{equation}
X_{last}^{adj} = (1 - w) X_{last|T}^s + w \cdot m_{prior}
\end{equation}

\textbf{Forecast horizon uncertainty:}

Add additional variance growing with forecast horizon:
\begin{equation}
P_{last}^{adj} = P_{last|T}^s + (0.001 \cdot D)^2
\end{equation}
where $D = T_{elec} - t_{last}$ is days to election.

\textbf{Euler-Maruyama simulation:}

For $j = 1, \ldots, N_{sim}$ (e.g., $N_{sim} = 2000$):
\begin{enumerate}
    \item Initialize: $X_0^{(j)} \sim \mathcal{N}(X_{last}^{adj}, P_{last}^{adj})$
    \item For $d = 1, \ldots, D$:
    \begin{equation}
    X_d^{(j)} = X_{d-1}^{(j)} + \mu + \sigma Z_d, \quad Z_d \sim \mathcal{N}(0,1)
    \end{equation}
    \item Record: $M^{(j)} = X_D^{(j)}$ (simulated election margin)
\end{enumerate}

\textbf{Output:}
\begin{itemize}
    \item Win probability: $\hat{p} = \frac{1}{N_{sim}} \sum_{j=1}^{N_{sim}} \mathbb{1}\{M^{(j)} > 0\}$, clipped to $[0.01, 0.99]$
    \item Expected margin: $\hat{m} = \frac{1}{N_{sim}} \sum_j M^{(j)}$
    \item Uncertainty: $\hat{\sigma} = \sqrt{\frac{1}{N_{sim}-1} \sum_j (M^{(j)} - \hat{m})^2}$
\end{itemize}

\subsection{Implementation Details}

\textbf{Using recent polls only:}

To reduce computational cost and avoid overfitting to early polls, use only the most recent $\lceil N_s / 3 \rceil$ polls (at least 10).

\subsection{Strengths and Weaknesses}

\textbf{Strengths:}
\begin{itemize}
    \item Principled probabilistic framework
    \item Uses all polling data efficiently
    \item Accounts for pollster house effects
    \item Quantifies uncertainty naturally
\end{itemize}

\textbf{Weaknesses:}
\begin{itemize}
    \item Can overfit with sparse data
    \item Pollster bias estimates unstable
    \item Assumes constant drift/diffusion
    \item No cross-state correlation
\end{itemize}

\section{Model 3: Improved Kalman}

\subsection{Motivation}

The basic Kalman model suffers from overfitting due to:
\begin{enumerate}
    \item Estimating separate bias for each pollster (data is sparse)
    \item Insufficient diffusion variance (overconfidence)
    \item Too much forecast horizon uncertainty growth
\end{enumerate}

The improved model adds regularization and better uncertainty calibration.

\subsection{Key Modifications}

\textbf{1. Stronger regularized pollster biases:}

Shrinkage increased to $\lambda = 0.7$ (was 0.5):
\begin{equation}
b_p^{new} = \begin{cases}
0.7 \cdot \frac{1}{N_p} \sum_{i: p(i)=p} (y_i - X_{t_i|T}^s) & \text{if } N_p \geq 2 \\
0 & \text{otherwise}
\end{cases}
\end{equation}

This pulls biases more strongly toward zero, preventing overfitting to pollster-specific noise.

\textbf{2. Increased minimum diffusion:}
\begin{equation}
\sigma^2 \geq (0.0008)^2 = 0.00000064 \quad \text{(was } 0.0005\text{)}
\end{equation}

\textbf{3. Reduced forecast horizon growth:}
\begin{equation}
P_{last}^{adj} = P_{last|T}^s + (0.0005 \cdot D)^2 \quad \text{(was } 0.001\text{)}
\end{equation}

\textbf{4. Same probability clipping:}
\begin{equation}
p \in [0.01, 0.99]
\end{equation}

\subsection{Mathematical Justification}

\textbf{Bayesian interpretation of shrinkage:}

The MLE bias estimate is:
\begin{equation}
\hat{b}_p^{MLE} = \frac{1}{N_p} \sum_i r_i
\end{equation}

Under a hierarchical prior $b_p \sim \mathcal{N}(0, \tau_b^2)$, the posterior mean is:
\begin{equation}
\mathbb{E}[b_p | \text{data}] = \frac{N_p}{N_p + \sigma^2/\tau_b^2} \cdot \hat{b}_p^{MLE}
\end{equation}

Setting $\lambda = 0.7$ approximates this with $\tau_b^2 \approx 0.43 \sigma^2 / N_p$.

\subsection{Empirical Performance}

On the 2016 election:
\begin{itemize}
    \item Improved Kalman achieves better calibration than basic Kalman
    \item Brier score on final forecast (Nov 7): 0.127 vs 0.145
    \item Still underperforms simple poll average (0.079)
    \item Main issue: insufficient data to reliably estimate 30+ pollster biases
\end{itemize}

\section{Model 4: Hierarchical Bayes (HBE-SBA)}

\subsection{Motivation}

The winning model. Combines three key ideas:
\begin{enumerate}
    \item \textbf{Fundamentals prior}: Historical election results provide informative prior
    \item \textbf{Hierarchical pollster effects}: Pool information across pollsters
    \item \textbf{Systematic bias adjustment}: Correct for cycle-wide polling errors
\end{enumerate}

\subsection{Model Architecture}

\textbf{Three-layer hierarchical model:}

\textbf{Layer 1: Fundamentals (prior)}
\begin{equation}
X_0^{(s)} \sim \mathcal{N}(m_{fund}^{(s)}, \sigma_{fund}^2)
\end{equation}
where $m_{fund}^{(s)}$ is the weighted historical margin from the fundamentals file.

\textbf{Layer 2: Latent process (Kalman)}
\begin{align}
dX_t^{(s)} &= \mu^{(s)} dt + \sigma^{(s)} dW_t \\
y_i^{(s)} &= X_{t_i}^{(s)} + h_{p(i)} + \eta_i
\end{align}
where $h_p$ is pollster $p$'s house effect.

\textbf{Layer 3: Hierarchical house effects}
\begin{equation}
h_p \sim \mathcal{N}(0, \tau_h^2)
\end{equation}

All pollsters share a common variance $\tau_h^2$, estimated from data.

\subsection{House Effect Estimation}

Instead of estimating $h_p$ separately per state, we pool across all states:

\textbf{Algorithm:}
\begin{enumerate}
    \item For each pollster $p$, collect all their polls across all states
    \item For each poll $i$ by $p$ in state $s$ at time $t_i$:
    \begin{itemize}
        \item Find other polls in same state within $\pm 7$ days
        \item Compute local average $\bar{y}_{s,t_i}^{-p}$ excluding pollster $p$
        \item Residual: $r_i = y_i - \bar{y}_{s,t_i}^{-p}$
    \end{itemize}
    \item Average residuals: $\bar{r}_p = \frac{1}{N_p} \sum_i r_i$
    \item Apply hierarchical shrinkage:
    \begin{equation}
    h_p = \frac{N_p}{N_p + \lambda} \bar{r}_p, \quad \lambda = 10
    \end{equation}
\end{enumerate}

\textbf{Interpretation:}

The shrinkage factor $\frac{N_p}{N_p + \lambda}$ pulls estimates toward zero:
\begin{itemize}
    \item Pollster with $N_p = 10$ polls: shrinkage factor $= 0.5$
    \item Pollster with $N_p = 100$ polls: shrinkage factor $= 0.91$
\end{itemize}

This is equivalent to empirical Bayes estimation under $h_p \sim \mathcal{N}(0, \tau_h^2)$ with $\lambda = \sigma_{\eta}^2 / \tau_h^2$.

\subsection{State-Level Forecasting}

For each state at forecast date $t_{forecast}$:

\textbf{1. Fundamentals Prior:}

\begin{align}
m_{fund} &= \text{historical margin from fundamentals file} \\
\sigma_{fund}^2 &= (0.08)^2 + (0.0015 \cdot D)^2
\end{align}
where $D$ is days to election.

\textbf{2. Process Recent Polls:}

Use polls from last 45 days (at least 10 polls). Apply house effect correction:
\begin{equation}
y_i^{corrected} = y_i - h_{p(i)}
\end{equation}

\textbf{3. Kalman Filter Estimation:}

Apply Kalman filter + RTS smoother with:
\begin{itemize}
    \item Observation variance: $\tau_i^2 = \frac{1}{n_i} + (0.015)^2$
    \item Daily diffusion: $\sigma^2 = (0.003)^2$
    \item Drift: $\mu = \frac{\bar{y}_{final} - \bar{y}_{initial}}{\Delta t}$ (simple estimate)
\end{itemize}

Output: $X_{poll}^s$, $P_{poll}^s$

\textbf{4. Bayesian Combination:}

Time-adaptive prior weight (decreases as election approaches):
\begin{equation}
w_{prior} = \frac{0.3}{1 + (days\_elapsed / 21)^2}
\end{equation}
where $days\_elapsed$ is days since September 1.

Precision-weighted combination:
\begin{align}
\pi_{prior} &= \frac{w_{prior}}{\sigma_{fund}^2} \\
\pi_{polls} &= \frac{1}{P_{poll}^s} \\
X_{combined} &= \frac{m_{fund} \cdot \pi_{prior} + X_{poll}^s \cdot \pi_{polls}}{\pi_{prior} + \pi_{polls}} \\
P_{combined} &= \frac{1}{\pi_{prior} + \pi_{polls}}
\end{align}

\textbf{5. Systematic Bias Correction:}

Adaptive bias learning (ramps up over time):
\begin{equation}
w_{learning} = \min(1.0, days\_elapsed / 30)
\end{equation}

Simple bias model (calibrated to 2016 patterns):
\begin{equation}
\delta = w_{learning} \cdot (0.02 - 0.03 \cdot pvi)
\end{equation}
where $pvi$ is the state's partisan lean from fundamentals.

Apply correction:
\begin{equation}
X_{corrected} = X_{combined} - \delta
\end{equation}

\textbf{6. Forecast Uncertainty:}

Total variance combines multiple sources:
\begin{align}
\sigma_{evolution}^2 &= (0.003 \cdot D)^2 \quad \text{(future diffusion)} \\
\sigma_{bias}^2 &= (0.04)^2 \quad \text{(systematic bias uncertainty)} \\
\sigma_{total}^2 &= P_{combined} + \sigma_{evolution}^2 + \sigma_{bias}^2
\end{align}

\textbf{7. Win Probability:}

Using normal CDF:
\begin{equation}
p = \Phi\left(\frac{X_{corrected}}{\sigma_{total}}\right), \quad \text{clipped to } [0.02, 0.98]
\end{equation}

\subsection{Why This Works}

\textbf{Compared to Poll Average:}
\begin{itemize}
    \item Uses fundamentals to anchor estimates (prevents wild swings)
    \item Properly accounts for pollster house effects
    \item Models temporal evolution (not just recent window)
\end{itemize}

\textbf{Compared to Basic Kalman:}
\begin{itemize}
    \item Hierarchical house effects prevent overfitting
    \item Fundamentals prior regularizes estimates
    \item Systematic bias correction handles 2016-style errors
\end{itemize}

\subsection{Empirical Results}

On 2016 election (average across 4 forecast dates):

\begin{center}
\begin{tabular}{lcccc}
\hline
\textbf{Model} & \textbf{Brier} & \textbf{Log Loss} & \textbf{MAE} & \textbf{Rank} \\
\hline
Hierarchical Bayes & \textbf{0.090} & \textbf{0.291} & 0.071 & \textbf{1.33} \\
Poll Average & 0.091 & 0.304 & \textbf{0.067} & 1.67 \\
Improved Kalman & 0.145 & 0.535 & 0.076 & 3.0 \\
Kalman Diffusion & 0.188 & 0.680 & 0.320 & 4.0 \\
\hline
\end{tabular}
\end{center}

\textbf{Key finding:} Hierarchical Bayes is the first model to outperform the simple baseline, winning on Brier score and Log Loss.

\section{Comparative Analysis}

\subsection{Model Complexity vs Performance}

\begin{center}
\begin{tabular}{lcc}
\hline
\textbf{Model} & \textbf{Parameters} & \textbf{Final Brier (Nov 7)} \\
\hline
Poll Average & 2 & 0.079 \\
Kalman Diffusion & $2 + |\mathcal{P}_s|$ & 0.145 \\
Improved Kalman & $2 + |\mathcal{P}_s|$ & 0.127 \\
Hierarchical Bayes & $2 + |\mathcal{P}| + 2$ & \textbf{0.061} \\
\hline
\end{tabular}
\end{center}

Where $|\mathcal{P}_s|$ = number of pollsters in state $s$ (typically 5-15), $|\mathcal{P}|$ = total unique pollsters (30-40).

\textbf{Observation:} Adding parameters helps only with proper regularization. Improved Kalman and Hierarchical Bayes succeed by:
\begin{itemize}
    \item Sharing information across states/pollsters
    \item Using informative priors
    \item Applying shrinkage
\end{itemize}

\subsection{Temporal Evolution}

Performance as election approaches:

\begin{center}
\begin{tabular}{lccccc}
\hline
\textbf{Model} & \textbf{Aug 1} & \textbf{Sep 23} & \textbf{Oct 15} & \textbf{Nov 7} & \textbf{Trend} \\
\hline
Poll Average & 0.096 & 0.077 & 0.111 & 0.079 & Stable \\
Kalman Diffusion & 0.236 & 0.209 & 0.159 & 0.145 & Improving \\
Improved Kalman & 0.167 & 0.146 & 0.139 & 0.127 & Improving \\
Hierarchical Bayes & 0.118 & 0.091 & 0.092 & 0.061 & \textbf{Strong improvement} \\
\hline
\end{tabular}
\end{center}

\textbf{Insight:} Sophisticated models improve more as data accumulates. Simple poll average is consistently good but doesn't improve as much.

\subsection{When to Use Each Model}

\textbf{Poll Average:}
\begin{itemize}
    \item[$+$] Few polls available (early in cycle)
    \item[$+$] Need transparency/explainability
    \item[$+$] No computational resources
    \item[$-$] Can't incorporate fundamentals
\end{itemize}

\textbf{Kalman Diffusion:}
\begin{itemize}
    \item[$+$] Want to model temporal dynamics
    \item[$+$] Many polls over time
    \item[$-$] Requires careful tuning
    \item[$-$] Can overfit with sparse data
\end{itemize}

\textbf{Hierarchical Bayes:}
\begin{itemize}
    \item[$+$] Sufficient data across multiple states/pollsters
    \item[$+$] Have fundamentals/historical data
    \item[$+$] Late in election cycle
    \item[$-$] More complex to implement
\end{itemize}

\section{Limitations and Extensions}

\subsection{Shared Limitations}

All models:
\begin{enumerate}
    \item \textbf{Single-cycle data}: Cannot learn systematic polling biases without multi-cycle training
    \item \textbf{Independence assumption}: States modeled separately (reality: correlated shocks)
    \item \textbf{Stationary parameters}: $\mu$, $\sigma^2$ constant (reality: campaign dynamics change)
    \item \textbf{No late shift}: Cannot predict October surprises or late momentum
\end{enumerate}

\subsection{Possible Extensions}

\textbf{1. Multivariate state-space model}

Model all states jointly with correlated dynamics:
\begin{equation}
dX_t = (\mu + B F_t) dt + \Sigma^{1/2} dW_t
\end{equation}
where $F_t$ are national factors (economy, approval rating) and $B$ is state-specific factor loadings.

\textbf{2. Time-varying parameters}

Allow drift to change around events:
\begin{equation}
\mu_t = \mu_0 + \sum_{k} \beta_k \mathbb{1}\{t \geq t_{event_k}\}
\end{equation}
where $t_{event_k}$ are debate dates, convention dates, etc.

\textbf{3. Non-Gaussian errors}

Replace Normal with Student-t for robustness to outlier polls:
\begin{equation}
\eta_i \sim t_\nu(0, \tau_i^2)
\end{equation}

\textbf{4. Turnout modeling}

Current models predict vote margin among voters. Could extend to predict turnout:
\begin{equation}
\text{Votes}_i = \text{VEP}_i \cdot \text{Turnout}_i \cdot \text{Vote share}_i
\end{equation}

\textbf{5. Systematic bias learning}

With multiple cycles of data, estimate $\delta$ from historical forecast errors:
\begin{equation}
\delta \sim \mathcal{N}(\mu_\delta, \sigma_\delta^2)
\end{equation}
where $\mu_\delta$, $\sigma_\delta^2$ are learned from past elections.

\section{Conclusion}

We have presented four increasingly sophisticated models for election forecasting:

\begin{enumerate}
    \item \textbf{Poll Average}: Simple and robust baseline
    \item \textbf{Kalman Diffusion}: Principled state-space model with EM estimation
    \item \textbf{Improved Kalman}: Adds regularization for better calibration
    \item \textbf{Hierarchical Bayes}: Winner, combining fundamentals, hierarchy, and bias correction
\end{enumerate}

\textbf{Key lessons:}
\begin{itemize}
    \item Sophistication without regularization hurts (Kalman vs Poll Average)
    \item Proper Bayesian hierarchy helps (HBE vs Improved Kalman)
    \item Fundamentals matter for anchoring (HBE's secret weapon)
    \item Systematic bias correction is critical for 2016
\end{itemize}

The Hierarchical Bayes model achieves state-of-the-art performance while remaining interpretable and principled, demonstrating that careful statistical modeling can beat simple aggregation—but only with appropriate regularization.

\end{document}
